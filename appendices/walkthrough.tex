\chapter{Memsource mobile app walkthrough}

Log in with the following credentials: username “VNovak” and password (removed). You will be presented with the list of all your projects which are presented in different tabs.

Take a look around and familiarize yourself with the screen.

When you’re ready, create a new translation project with the following parameters: 

\begin{itemize}
\item project name: “yet another project”
\item Client: Slavia
\item Domain: Machinery
\item Source language: English 
\item Target languages: Czech, German
\item Due date: 19th of January at 7pm
\item Workflow steps: select translation and revision
\end{itemize}




Verify the created project and try to list jobs of the project. The list should be empty at this time. 

Now let’s create a new job in the project. The job should be created from a file named “important document.doc” which is present in the device's Google Drive found under “mobile app/testfiles”. 
Keep the job’s target languages se to “cs” and “de”.
Set the job’s due date to 11th of January, 10 am and select the “VNovak23“ user as the linguist for both languages. 
Make sure that comments and hidden text from the word document are imported and then create the job. 

Wait for the job to be imported and then list the jobs and switch to the “Revision” workflow step. Select the job part whose target language is German and change its due date to 17.1.2017 11am and save. Verify the due date is updated.

The next task is to add existing translation memories to the project. If you list the translation memories, you will see that none are assigned yet.  Attach the “Software TM“ translation memory to the German target language. Use the filter if you’re having trouble finding it. Attach it with read and write modes enabled, a penalty of 10% and all workflow steps. For Czech, attach “interview” TM only in read mode and “HackerX TM” in both read and write, also for all workflow steps. Verify the translation memories are attached after saving. 

The next step is to add term bases. 
Attach the “Clock Industry” term base with read, write and QA enabled. Also add the “jim” term base in read mode. 

After the term bases are attached, go back to the project list screen and try to use the search feature to find the project you just created. Verify it shows up in the search results and that you can open it. 

That's it for today, thank you for your participation! 