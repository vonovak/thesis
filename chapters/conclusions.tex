\chapter{Conclusions and Future Work}

Mobile devices are ubiquitous and people use them extensively in their everyday lives. As a consequence, many online services are accessible not only through the web, but also via applications made specifically for mobile devices. Indeed, the presence of mobile devices is so strong that many products are even built solely for mobile and do not exist on the desktop.

Memsource Cloud is an online service that had no mobile application and the goal of this thesis was to change it. In its scope I have introduced Memsource to the reader and explained the motivation. Since it was clear from the beginning that the application must be multiplatform, I have performed an analysis and comparison of tools for multiplatform development. Prior to starting the development I have collected requirements for the application and created mockups at different fidelity levels. The application was implemented mostly in JavaScript and React, with Swift, Objective C and Java being used in the functionality implemented natively. 

I have laid the foundations for an application that will allow Memsource users access the most important features of Memsource Cloud. The current functionality includes CRUD operations on projects and jobs, as well as working with term bases and translation memories and other features, as described by the requirements in section \ref{sec:requirements}.

What I have not managed to implement, are features for linguists because it turned out that an API for project listing given different categories, which I originally thought was a very simple matter, was not available. The API is to be developed and adding the features for linguists shall be simple, as they are already implemented for the project manager role.

React Native, despite its young age proved to be a valuable and functional library for multiplatform mobile app development, albeit sometimes it required a little more work than I would prefer to make things work exactly the way I wanted. Its easy integration with the underlying platform was important for implementing the native modules. Also MobX is a library that works very well with React and I have enjoyed working with.


The future work will involve further development of the application within Memsource. I will continue adding more features to the application, and improve the styling of some components. Another important future step is to introduce a continuous integration tool.