\chapter{Conclusions and Future Work}

Mobile devices are ubiquitous and people use them extensively in their everyday lives. As a consequence, many online services are accessible not only through the web, but also via applications made specifically for mobile devices. Indeed, the presence of mobile devices is so strong that many products are even built solely for mobile and do not exist on the desktop.

Memsource Cloud is an online service that had no mobile application and there was a desire to change it. In the scope of this thesis I have introduced Memsource to the reader and explained the motivation. Since it was clear from the beginning that the application must be multiplatform, I have performed an analysis and comparison of the tools for multiplatform development. Prior to starting the development I have collected requirements for the application and created mockups at different fidelity levels. The application was implemented mostly in JavaScript and React, with Objective C, Swift and Java being used in the functionality implemented natively. 


I have laid the foundations for an application that will allow Memsource users access the most important features of Memsource Cloud. The current functionality includes CRUD operations on projects and jobs, as well as working with term bases and translation memories and other features such as search or multi-user support.

React Native, despite its young age proved to be a valuable and functional tool for multiplatform mobile app development. Also MobX is a library that works very well with React and I have enjoyed working with.


The future work will involve further development of the application within Memsource. There is a severe bug in the Android version of the app caused by a bug in React Native core. Rather than waiting for the bug to be fixed in React Native, I have decided for a workaround (to replace the Android navigation bar component) which will also have the pleasant side effect of improving navigation bar code structure in the application. The downside is that the app will not be using the native navigation bar provided by Android. I will, of course, continue adding more features to the application, and improve the styling of some components. Other important future step is to introduce a continuous integration tool.