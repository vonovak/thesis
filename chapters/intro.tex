
\chapter{Introduction}

Memsource is a Prague-based company that develops an online platform for translation, translation management and analytics called Memsource Cloud \footnote{https://www.memsource.com/en/features}. The platform is provided as software as a service (SaaS) and uses a freemium business model. It is used by translation agencies as well as freelancers to administer their projects, the documents they need to translate and provides an editor tailored specifically for translation needs. It enables its customers to keep all important information in one place and increase their productivity. In the document, I will use the names Memsource and Memsource Cloud interchangeably and will make a note in case we need to distinguish between them.

To start using Memsource, user has to sign up and choose one of the offered plans. After completing the registration process, they can start using the project management and translation tools. The management part consists of a web-based interface where the user can administer their translation projects, jobs (translated documents), translation memories and term bases (these terms will be explained later on), users and other features. Closely related is the Memsource editor which is available both for major web browsers and as a standalone application. The editor serves as a specialized tool for performing the translation and includes features for improving the speed and quality of translation. The changes made in either of the editors are synchronized. One of the newer features is the analytics bundle which allows the user to see translation progress and performance and thus get a deeper insight into their business.

Memsource is designed to support three kinds of customer groups: individual translators, translation agencies and translation buyers and offers corresponding versions, sub-editions and services to each of those.
The Translator edition is meant for individual translators or freelancers and has only a basic set of features offered free of charge. The edition for Translation Agencies adds more features on top of the freelancer version, notably the possibility to work with users, set up user roles and workflows. This allows project managers within the agency to distribute translation jobs among translators and specify the workflow through which multiple versions of a translated document can be kept in a project. A typical workflow may consist of translation, editing and proofreading.

The ultimate editions also support advanced features and, more importantly, access to Memsource Cloud API. Lastly, the version for Translation Buyers is intended for corporate customers who need to have various texts translated for their business and through Memsource, they're connected to the translation agencies or freelancers who will do the job for them. 


\section{Motivation}

Memsource operates on the market of CAT (computer-aided translation) tools since 2010 an it is its best effort to provide modern and innovative solutions for translators based on its SaaS model. This effort is fulfilled by a set of described web and desktop-based applications. The current tools developed by Memsource are designed for use on computers mostly with a large screen, i.e. laptops, desktops, or large tablets through a web browser (with the exception of desktop editor which can be installed on Windows, OS X and Ubuntu).

The sector of mobile devices, however, remains largely uncovered by Memsource. In today's world, mobile devices play an ever important role, allowing people to access online resources from virtually anywhere and at any time. There are a number of studies that show the increasing presence of mobile devices on the internet and in both our professional and personal lives. 
Statista  offers an overview of Smartphone share of visits to websites in the United States in 2014 and 2015, by industry \cite{statista}. This statistics shows that the shares vary greatly between industry, with technology websites having 11.7\% share of visits from mobile, while for media and entertainment, mobile accounts for 36.6\%. Comscore goes even further and in its study from 2014 \cite{comscore}, it claims that more than a half of time spent with digital media (social networks, videos, magazines, etc.) in the U.S. is spent on mobile devices. An interesting blogpost by Google AdWords Vice President from 2015 \cite{googleAdwords} reads that ``more Google searches take place on mobile devices than on computers in 10 countries including the US and Japan.''. It clearly follows we have to design our software products to play well with mobile devices and the limitations that are inherent to them. 

While Memsource can be accessed from a mobile device through its internet browser, the web browser cannot take full advantage of the features of the platform it runs on. More specifically, with just the web browser, it is not possible to upload files for translation directly from email inbox, which is one of the main channels through which translation inquires are made. Making the project and job creation from email as smooth as possible is one the most important goals that the application should allow and that are overly complicated or impossible with a standard web browser.

Creating a mobile application therefore gives us more flexibility. Apart from that, on the Memsource Cloud web, there are a number of options and settings available (some of which are not used frequently because they support advanced functionality) and in some cases, this makes the UI quite complex. Rendering such UI on a mobile device's browser results in inconsistencies as well as limited usability even though it is programmed responsively.

Memsource, as a leading translation platform provider wants to be able to provide its core features accessible to customers who are on the go or do not have a computer at their disposal. For the aforementioned reasons, the existing solution is not suitable for such needs and therefore, the goal of this thesis is to develop an client application specifically tailored for use on mobile devices. This app should contain a subset of the features that are currently available and make the simple to use, keeping in mind the specifics of development for mobile devices.




